Inhaltsverzeichniss:

\section{Einleitung}

	1.1 Ziel und Relevvanz der ARbeit

	1.2 Grundlagen der Thermografie

	1.3 Definition der Schluesselbegriffe

2. Anwendungen der Thermografie

	2.1 Allgemeine Anwendungen
		
		Die Anwendungen der Thermografie erstrecken sich über fast alle Bereiche der Industrie.
		- Wärmedämmung, Luftdichtheit Bau
		- Feuerwehr
		- Geologen mit Sateliten Bilder, Ozean etc
		- Polizeit, Militöär
		-Sportwissenschaft, Durchblutung
		- Zerstörungsfreie Werkstoffprüfung
		- Photovaltaik QS Kontrolle
	
	2.2 Nachteile, Grenzen des Verfahren
		- teuer
		- Emssisionsfaktor muss bekannt sein, ändert sich stark bei Metallen
		-reflexionen bei Metallen stören stark
		- Genauigkeit verähltnismä0ßug schlecht
		- Witterungsverhätlnisse (wind Sonne, feuchtigkeit) verändert Genauigkeit
		- nur Oberfläche
	
	
	
	2.3 Vorteile der berührungslosen Temperaturmessung
		
		- große Fläche gleichzeitig
		- Berührungslos, größerer Entfernung
		- Umgebung egal (elektrisch, radioaktiv, aggresives Medium
		- Dunkjelheit nicht auschlaggebend
		- großer Temperaturbereich
		- schnell, bewegende Objekte kein Problem
		- kein Verschleiss
			- keine mechanische Beschädigung
	
	2.3 Praxisbeispiel bei Widerstandsmessung von SPulen
		
		- allgemeine AUswahlkriterien
			Tempbereich
			Umbegungstemp
			Messfleckgröße
			Abstand zur Messung
			Material, Oberflöche
			Ansprechzeit IR
			Schnitstelle
		Verwendet: optris CT laser LT (SPektralbereich 8-14 mycrometer) Schärfepunkt CF2 bei 200mm 2.75mm Fleck mit Ziellaser
		Ausgänge analog 0/4 - 20 mA oder 0-5/10 V,   optional USB, RS232, RS485, CAn Profibus DP, Ethernet

			angeshclossen über ProfibusDP
	
		
	
	
3, Problematik, Fehlerauswertung


4. Kritische REflexion

5. Fazit