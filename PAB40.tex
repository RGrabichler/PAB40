Inhaltsverzeichniss:

\section{Einleitung}

	1.1 Ziel und Relevvanz der ARbeit

	1.2 Grundlagen der Thermografie

	1.3 Definition der Schluesselbegriffe

2. Anwendungen der Thermografie

	2.1 Allgemeine Anwendungen
		
		Die Anwendungen der Thermografie erstrecken sich über fast alle Bereiche der Industrie.
		- Wärmedämmung, Luftdichtheit Bau
		- Feuerwehr
		- Geologen mit Sateliten Bilder, Ozean etc
		- Polizeit, Militöär
		-Sportwissenschaft, Durchblutung
		- Zerstörungsfreie Werkstoffprüfung
		- Photovaltaik QS Kontrolle
	
	2.2 Nachteile, Grenzen des Verfahren
		- teuer
		- Emssisionsfaktor muss bekannt sein, ändert sich stark bei Metallen
		-reflexionen bei Metallen stören stark
		- Genauigkeit verähltnismä0ßug schlecht
		- Witterungsverhätlnisse (wind Sonne, feuchtigkeit) verändert Genauigkeit
		- nur Oberfläche
	
	
	
	2.3 Vorteile der berührungslosen Temperaturmessung
		
		- große Fläche gleichzeitig
		- Berührungslos, größerer Entfernung
		- Umgebung egal (elektrisch, radioaktiv, aggresives Medium
		- Dunkjelheit nicht auschlaggebend
		- großer Temperaturbereich
		- schnell, bewegende Objekte kein Problem
		- kein Verschleiss
			- keine mechanische Beschädigung
	
	2.3 Praxisbeispiel bei Widerstandsmessung von SPulen
		
		- allgemeine AUswahlkriterien
			Tempbereich
			Umbegungstemp
			Messfleckgröße
			Abstand zur Messung
			Material, Oberflöche
			Ansprechzeit IR
			Schnitstelle
		Verwendet: optris CT laser LT (SPektralbereich 8-14 mycrometer) Schärfepunkt CF2 bei 200mm 2.75mm Fleck mit Ziellaser
		Ausgänge analog 0/4 - 20 mA oder 0-5/10 V,   optional USB, RS232, RS485, CAn Profibus DP, Ethernet

			angeshclossen über ProfibusDP
	
	
	2.3.1  Verwendetes IR Messgerät
		Im beschriebenen Messsytem wird das CT laser LT System von optris benutzt.  
		Vorteile sind sehr kleine Messflecken, ein Doppel-Laservisier zur Messfeldmarkierung und eine 75:1 Optik Scharfeinstellung. Die Reaktionszeit 
		beträgt 9 ms und ist damit hervorragend für schnelle Qualitiätsmessungen geeignet. Die Anbindung an die SPS  erfolgt über einen 4-20mA Analogausgang, 
		diverse Feldbus Ansteuerungen sind zusätlzich möglich, im vorliegenden Fall aber nicht benutzt. Weitere Parameter sind:
		-Scharfpunkt Einstellung bei CF2 Modell : 1.9mm @ 150mm
		- Sytemgenauigkeit +- 1%
		- Emissionsgrad 0.1 - 1.1
		Transmissionsgrad 0.1 - 1.0
		-Signalverarbeitung mit MAX-, Minwerthaltu ng, Mittelwert, Hysterese
		-Visierlaser mit 1mW, Wellenlänge 635nm
		
	2.3.2 Temperaturskompensation über Polynom
		Grundsätölzich dient das IR Gerät nicht als eigensätniges QS System, sondern als 
		Hilfsmittel, womit der Widerstand der SPule bestimmmt werden kann. Verschiedene Temperaturen des Kupferdrahts 
		äußern sich in abweichenden Widerständen, bezogen auf 20Grad. Problem ist jedoch, dass die Temepraturverteilung nicht
		homogen ist, und somit nicht einfach die Temp. des Kupfers ermittel werden kann. Dadurch kann mit dem Temperaturkoeffizienten (0.00393) von Kupferdrahts
		und der Formel DeltaR = alhpa + DeltaT +* Rk, bzw. Rw = Rk + deltaR nicht einfach der Widerstand ermittelt werden. 
		Abhilfe schafft ein Verfahren zur Bestimmung eines AUsgleichspolynoms, dass zu ermittelten Prüfwerten
		ein Poylnom liefert. Das ermitteln dieser Werte wird in einem eigenem Prüfprogramm erledigt. Dazu wird eine umspirtzte Spule in die Prüfanlage gefahren.
		Im Zyklus von 5s wird jeweils ein Wertepaar, bestehend aus Temp und WIderstand ermittelt und in eine Tabelle geschrieben.
		Das Polynom wird mit der mathematischen Verfahren "Methode der kleinsten Quadrate" bestiummmt.
		Ein extra dafür erstelltes Python Skript liest aus der Excel Datei die zuvor ermittelten Wertepaare aus und 
		berechnet automatisch die passenden KOeffizienten. Im folgenden Abschnitt wird kurz auf das Skript eingegangen.
	
		Das Modul openpyxl ist eine Bibliothek, die das Lesen und Schreiben in Excel Dateien erlaubt.
		Mit dem Befehl py.load_workbook... in Zeile 23 wird die Excel mappe geladen und der Variable wb für WOrkbook zugewiesen.
		In der FOr Schleife ziwcshen den Zeilen 77 und 92 ewrden die Werte für den Widerstand und die Temperatur eingelesen. 
		Mit der Zeile 101 wird aus jedem berechneten Mittelwert ein Temperaturkoeffizient gebildet. Mittels Temperatur und Koeffizient können
		die Polynomparameter ermittelt werden. Die Codezeile 136 zeigt den erforderlichen Befehl Polyfit(x[i], y[i], degree) aus der Lib numpy. 
		Die Ausgabe erfolgt in der Python Shell und produziert folgende Werte für ein Polynom 4ter Ordnung:
		\image Poly
		Der Vergliech mit dem WIderstandswert bei 20grad (29.28) ergibt eine max. Abweichung von 0.22 Ohm. 
		Im Serienbetrieb kann aufgrund dieser ermittelten Polynomparameter die Steuerung die Spulen mit unterschiedl8ichen WIderstandswerten 
		eine sichere Prüfung, bezogen auf 20 Grad, gewährleisten. Somit sind die Werte aussagekräftig undv vergleichbar.
	
3, Problematik, Fehlerauswertung


4. Kritische REflexion

5. Fazit